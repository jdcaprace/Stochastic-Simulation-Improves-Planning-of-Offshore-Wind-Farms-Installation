\subsection{Aproach 2 -- Probabilistic workability}
The probabilistic approach involves computing of the monthly conditional probability of working. A method is triggered based on weather criteria, time window and associated probability. It returns a true or false value whether to proceed to the next activity or to wait until the next time frame of good weather. A loop iterates until the result is true which gives a green light to carry out the next activity (sailing, installing, transferring piles, etc.). The time elapsed until the iteration return a true result is considered as a waiting time. At each iteration of the DES, new random numbers are set-up which change the sequences of the binary values and therefore the complete lead time of the processes. Fig. \ref{fig:iterations} represent the lead time of the installation of 60 OWTs along 400 iterations. As the result is highly stochastic, testing the convergence of the output is required. Fig. \ref{fig:convergence} presents the convergence of the lead time related to the installation of 60 OWTs. It is observed that the accumulated mean value tend to converge roughly after 250 iteration (variation of less than one day per iteration

TODO: ADD paragraph to compare result with approach 1.