\section{Conclusions}
\label{conclusions}
%As the number of OWTs in offshore wind projects increases, and the OWFs are located further away from shore, there is a need to develop novel approaches to test and compare the performance of various intergation strategies before the implementation. %With the aid of simulation, different logistics strategies can be tested and compared before starting to implement a new concept in reality so that risks can be reduced.

%Various projects of OWFs are actually suffering considerable delays and cost overruns due to a bad evaluation of weather risks and restricted weather windows for logistic processes at sea, \cite{TCE12, ISOPE2013}. Logistics costs represent definitively a major risk factor for the offshore wind industry.

In this study, a comprehensive novel methodology of installation scheduling of OWFs is introduced towards minimum \textit{lead time}. Climate parameters, installation strategies, assembly processes, manufacturing constraints as well as transportation resources are simulated within the installation phase of an OWF. The results are demonstrated to support the decision making related to the installation and logistics strategy. The consequences of different decisions can be assessed and favorable solutions that minimize lead time can be selected. Better efficiency will be attained improving the predictability and transparency of the logistical processes, both in ports and at sea. In order to ascertain higher accuracy in the results, costs related to operating strategies of jack-up and transport vessels has to be investigated in details.

The contribution presented in this paper relates the development of a deterministic and probabilistic approach to consider the operability of the vessels at sea. In this context, it is identified that there is a good agreement between the two approaches. However, it has been demonstrated that the probabilistic approach may slightly underestimate the completion time of the projects.

The new approach may improve the planning and control of the logistics processes in the offshore industry reducing associated risks. It was found out that the effect of weather significantly influences the total project lead time. The analysis indicated that the starting months of the project as well as the number of OWTs to be installed are somehow related. In this context, for few numbers of OWTs (small wind farms of less than 45) it is advisable to start in March or early April. On the other hand, for large number of OWTs (large wind farms above 45), it is advisable to start the project in July or split the project into two phases. The lead time is also affected in a secondary order by the type of installation strategy implemented. In this regard single blade (SB) strategy reduce the lead time on average by 5 to 6\%.

As the number of OWTs in offshore projects increases and the OWFs are located further away from shore, there is a need to develop specialized vessels, transfer systems and installation strategies that will reduce the influences of rough sea conditions on the installation phases.

%Since the methodology relies both metocean time series and probablistic approach, the planning tool may be used during the operational phase of constructing or operating a wind farm. Thus it could be used as a day to day simulator in order to support real-time decisions. 


