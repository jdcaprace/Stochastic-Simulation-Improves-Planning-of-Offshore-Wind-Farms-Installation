\subsection{Two approaches}
Metocean time series of wind speed and wave height has been used in the simulation model to quantify the total completion time of an OWF including 60 OWTs. Wind speed was measured every 10 minutes at 10 meters of altitude in two perpendicular directions and wave height every 30 minutes. Data have been recorded from January 1995 and December 2008.
%In order to have coherence between the time intervals, wave height measurements has been linearly interpolated every 10 minutes.
Fig. \ref{fig:windwave} shows the mean ($\mu$) and standard deviation ($\sigma$) of wind speed and wave height time series.
The wind profile power law relationship, presented in equation \ref{eqn:windprofile}, is used to estimate the wind speed $u$ at height $z$, when $u_{r}$ is the known wind speed at a reference height $z_{r}$, \cite{Justus_1976, 1978Peterson}. The exponent $\alpha$ is an empirically derived coefficient that varies depending on the stability of the atmosphere, temperature, pressure, humidity, time of the day and terrain roughness. Here this coefficient is taken equal to $0.1$, value recommended over open water surfaces, \cite{WindEnergy2010}.

\begin{equation}
\label{eqn:windprofile}
u = u_{r} \left( \frac{z}{z_r} \right)^{\alpha}
\end{equation}

Two approaches have been considered in the DES. They are differing by the way to use the weather historical time series. In the first one (approach 1 -- Deterministic workability), weather data are directly used in the DES to assess if an activity can be started based on weather window and workability criterion while in the second one (approach 2 -- Probabilistic workability) probabilities to work are previously computed. Approach 1 has the advantage to perform assessments on historical data which allow the calibration of the simulation model based on past projects but does not allow the scheduling of new projects. Contrarely, approach 2 is designed to be used for planning of new projects as it is using working probabilities computed on monthly base. These probabilities to work have been computed for each activity considering the workability and weather window restrictions.

???? Here the average monthly workability has been implemented in the simulation model. 

If there are two activities to be carried out one after another without interruption, it requires the implementation of conditional probability.  
In probability theory, \cite{Thalemann2012}, a conditional probability measures the probability of an event given that another event has occurred. Following Kolmogorov definition, given two events A and B, the conditional probability of A given B is defined as the quotient of the probability of the joint of events A and B, and the probability of B, see equation \ref{eqn:kolmogorov}.

\begin{equation}
\label{eqn:kolmogorov}
P \left(A \vert B \right) = \frac{P \left(A \cap B \right)}{P \left( B \right)}
\end{equation}

The tree diagram shown in Fig. \ref{fig:condprob} may represent a series of independent events or conditional probabilities. Each node on the diagram represents an event and is associated with the probability of that event. The root node represents the certain event and therefore has probability 1. Each set of sibling nodes represents an exclusive and exhaustive partition of the parent event. The probability associated with a node is the chance of that event occurring after the parent event occurs. The probability that the series of events leading to a particular node will occur is equal to the product of that node and its parents' probabilities.

In this paper, the upper branch of the tree is used in order to compute the workability of two successive activities. For example, considering the pre-pilling phase, the probability of the transport ship to leave the port is first assessed (first activity), then, conditional probability to perform the transfer of piles onto the instllation vessel is performed (second activity).