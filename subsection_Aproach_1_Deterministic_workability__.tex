\subsection{Deterministic operability}

A new simulation run is created for each month from 1995 until to 2006. The simulation starting date of the project has been assumed to be the first day of each month. Thus, it represents 12 simulations per year what is mean 144 simulations in total. Therefore, the average lead time ($\mu$) and standard deviation ($\sigma$) is deduced from the 12 points available for each month.

Fig. \ref{fig:comparison} shows the influence of the starting date of the project on the mean lead time ($\mu$) of the installation of 60 OWTs measured in days between the start of pilling activity of the first OWT until the completion of an OWF containing 60 units. Shaded area shows the standard deviation limits ($\sigma$), i.e. $\mu + \sigma$ and $\mu - \sigma$. It is observed that the standard deviation is presenting a high variability along the year with the minimum value during the summer in August (50 days) and the maximum value at beginning of Autumn in October (94 days). That gives an insight how the weather plays a significant role in the offshore wind turbine installation operations.