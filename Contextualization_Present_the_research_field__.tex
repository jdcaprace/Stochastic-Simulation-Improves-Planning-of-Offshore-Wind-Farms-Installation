%Contextualization - Present the research field and show the importance of the main area. Make terms and processes familiar.  Evidencing recent research and findings. Recently there have been significant advances in ...
\textbf{Contextualization} --
Offshore Wind Farm (OWF) is an emerging technology in the wind energy conversion system. These wind resources are abundant, stronger, and are more consistent in terms of their availability than land-based wind resources. The average size of a grid connected OWF in 2014 was 368 MW, average water depth of wind farms completed or partially completed was 22.4 meters and the average distance to shore was 32.9 km, \cite{Giorgio2015}. The European Wind Energy Association expects that by 2020 offshore wind power will account for 4 to 4.2\% of Europe's energy demand with an installed capacity of 40 GW,  \cite{EWEA2011, Kaldellis2013}.

The potential of wind energy increases as one goes far from the coast line (30 to 100 kilometers), therefore implying deeper water depth (20 to 50 meters), higher power turbines ($\geq 5$ MW), and stronger foundations to support the turbine components. As a matter of fact significantly higher energy production is achieved due to larger wind turbine ratings and stronger wind profiles, \cite{Sun2012298}. Moreover, to be economically advantageous, plants have also to grow in size, i.e. $\sim600-1000$ new wind turbines per year, \cite{EWEA2011} and \cite{Kaldellis2013}. This will further complicate the logistic operations of the offshore wind energy systems, which requires special purpose vessels with a higher deck capacity to transport the components (turbine and foundation). At the same time, cranes with good lifting capacity should also be available in order to carry out the lifting and installation operations without compromising the safety of the crew on board. This implies serious financial, technical and logistical effort.

It explains why the power production from offshore wind is still significantly more expensive than power generation from onshore wind farms. Table \ref{tab:typicalcost} shows the typical cost breakdown of both, onshore and offshore.%, compiled and adapted from \cite{Henderson2003}, \cite{Junginger2004},  \cite{UK10}, \cite{TCE12}, \cite{IRENA12}, \cite{Voormolen_2016}.
As shown, most of the investment cost are related with acquisition of wind farms including wind turbines, electrical infrastructure and civil engineering work. Nevertheless, for OWFs, the installation and transport costs of an offshore wind energy plant is a significant contributor to the total initial-cost and likely reach 20\%. Therefore, the risks of over-costs and delays at the construction site, transport chain, production and storage should be carefully controlled.
