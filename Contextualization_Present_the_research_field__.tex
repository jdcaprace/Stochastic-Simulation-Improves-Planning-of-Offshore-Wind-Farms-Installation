%Contextualization - Present the research field and show the importance of the main area. Make terms and processes familiar.  Evidencing recent research and findings. Recently there have been significant advances in ...
\textbf{Contextualization} --
Offshore Wind Farm (OWF) is an emerging technology in the wind energy conversion system. These wind resources are abundant, stronger, and are more consistent in terms of their availability than land-based wind resources. The average size of a grid connected offshore wind farm in 2014 was 368 MW, average water depth of wind farms completed or partially completed was 22.4 meters and the average distance to shore was 32.9 km, \cite{Giorgio2015}. The European Wind Energy Association expects that by 2020 offshore wind power will account for 4 to 4.2\% of Europe's energy demand with an installed capacity of 40 GW, \cite{EWEA2011, Kaldellis2013}.

To be economically advantageous, plants have to grow in size ($\geq 5$ MW, i.e. $\sim600-1000$ new wind turbines per year). As a consequence, the entire process of production, mounting and operating offshore wind plants has to be optimized. As a matter of fact significantly higher energy production is achieved due to larger wind turbine ratings and stronger wind profiles, \cite{Sun2012298}. Numerous OWFs are commonly built in comparatively deep water (20 to 50 meters) and far distances (30 to 100 kilometers) from shore. Higher wind speeds, waves and the salty air contribute to the harmful environment at sea that significantly reduces the accessibility of an OWFs, \cite{Smit2007}.

Facing these constraints, OWFs have to grow in size in order to be profitable ($\geq 5$ MW, i.e. $\sim600-1000$ new wind turbines per year), \cite{EWEA2011} and \cite{Kaldellis2013}. This implies serious financial, technical and logistical effort. Due to the rise of the size of the offshore components, logistical challenges arise, which often induce immense costs.

Table \ref{tab:typicalcost} shows the typical cost breakdown of both, onshore and OWFs, compiled and adapted from \cite{Henderson2003}, \cite{Junginger2004}, \cite{UK10}, \cite{TCE12}, \cite{IRENA12}. As shown, most of the investment cost are related with acquisition of wind farms including wind turbines, electrical infrastructure and civil engineering work. Nevertheless, for OWFs, the installation and transport costs of an offshore wind energy plant can vary between 5\% to 30\% of the total cost and likely reach 20\%. Therefore, the risks of over-costs and delays at the construction site, transport chain, production and storage should be carefully controlled.
