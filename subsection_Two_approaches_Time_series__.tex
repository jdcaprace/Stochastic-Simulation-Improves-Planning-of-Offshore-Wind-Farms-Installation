\subsection{Two approaches}
Time series of wind speed and wave height has been used in the simulation model to quantify the total completion time of an OWF including 60 OWTs. Wind speed was measured every 10 minutes at 10 meters of altitude in two perpendicular directions and wave height every 30 minutes. Data have been recorded from January 1995 and December 2008.
%In order to have coherence between the time intervals, wave height measurements has been linearly interpolated every 10 minutes.
Fig. \ref{fig:windwave} shows the mean ($\mu$) and standard deviation ($\sigma$) of wind speed and wave height time series.
The wind profile power law relationship, presented in equation \ref{eqn:windprofile}, is used to estimate the wind speed $u$ at height $z$, when $u_{r}$ is the known wind speed at a reference height $z_{r}$, \cite{1978Peterson}. The exponent $\alpha$ is an empirically derived coefficient that varies dependent on the stability of the atmosphere. Here this coefficient is taken equal to $0.1$, valor recommended over open water surfaces.

\begin{equation}
\label{eqn:windprofile}
u = u_{r} \left( \frac{z}{z_r} \right)^{\alpha}
\end{equation}

Two approaches have been considered in the DES. They are differing by the way to use the weather historical time series. In the first one (approach 1), weather data are directly used in the DES while in the second one (approach 2) probabilities are previously computed. Approach 1 has the advantage to perform assessments on historical data which allow the calibration of the simulation model but does not allow the scheduling of new projects. Contrarely, approach 2 is designed to be used for planning of new projects as it is using probabilistic weather model.


The second approach is implementing the probabilistic approach where the working and non working probabilities have been computed on monthly base. Monthly probabilities of working and non working have been computed for each activity considering the workability and time window weather restrictions. Here the average monthly workability has been implemented in the simulation model. 

Each activity will have workability and time window restrictions and the first approach will compare these values with the real weather database. For example if the ship should leave the port or not, can be checked this way and it may have to wait at port until favorable weather exists. The second approach uses the same principle but instead of comparing the weather restriction with the data base, it utilizes the monthly percentage of probability for each activity. If there are two activities to be carried out one after another without interruption, it requires the implementation of conditional probability.  
In probability theory, (\cite{Thalemann2012}), a conditional probability measures the probability of an event given that (by assumption, presumption, assertion or evidence) another event has occurred. 

