\subsection{Aproach 2 -- Probabilistic}
The probabilistic approach involves computing the monthly probabilities of working and non working for each activity considering the workability and time window weather restrictions. It is highly affected by the random seed number and requires testing the convergence of the simulation runs. 
For instance, a method will be triggered and based on the weather restriction , time window and associated probability, it gives a value of 0 or 1 (yes/no) which is considered as a deciding factor whether to proceed to the next activity or to wait until good weather exists. The loop iterates until the result is 1 (yes) which gives a green light to carry out a certain activity (sailing, installing, transferring piles, etc.). The time elapsed until the iteration gives a result of “1” is considered as a waiting time. Changing the random stream number will change the sequence of the binary values and it will result in having different waiting times until it reaches a green light, thereby making the output lead time stochastic (refer Fig. xx). The Fig. xx presents a convergence test for a specific start date of the project and it is clear from the Fig. xx that the mean values tend to converge roughly after xxx iterations.