\subsection{Methodology}
The development of the tool has been started in order to give an answer to a well-defined industrial requirement: the need to evaluate the risk of the strategic decision related to the installation of an OWF before its construction. To answer this question, a DES have been developed to assess what is the influence of the planning disturbance parameters on the lead time and costs of OWFs installation. A validation case have been developed in partnership with an industrial that provided all the data coming from a real OWF.

The objective of any installation of OWFs is to reduce as much as possible the construction time and costs. The capacities of vessels, harbour facilities and manpower have to be chartered for the planned project time and each additional day leads to extra costs. Therefore the simulation model presented in this paper focuses on the risks analysis of all the production disturbances mentioned before.


Furthermore, due to the random nature of the process disturbances, it is not sufficient to rely on the results of a single run. In order to obtain useful results, it is necessary to simulate a scenario several times to identify extreme values and to be able to consider the influence of random variables. So that after running it for a relevant number of times, the user can integrate the output of different results and get the standard deviation diagram, for each Key Performance Indicator (KPI). Similarly, the overall process can be repeated for various starting date within the range of available weather data.

