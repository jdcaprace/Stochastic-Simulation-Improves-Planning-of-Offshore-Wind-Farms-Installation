\subsection{Methodology}
The development of the tool has been started in order to give an answer to a well-defined industrial requirement: the need to evaluate the risk of the strategic decision related to the installation of an OWF before its construction. To answer this question, a DES have been developed to assess what is the influence of the planning disturbance parameters on the lead time and costs of OWFs installation. A validation case have been developed in partnership with an industrial that provided all the data coming from a real OWF.

The objective of any installation of OWFs is to reduce as much as possible the construction time and costs. The capacities of vessels, harbour facilities and manpower have to be chartered for the planned project time and each additional day leads to extra costs. Therefore the simulation model presented in this paper focuses on the risks analysis of all the production disturbances mentioned before.

Fig. \ref{FigureSimulationModel} presents the DES workflow that includes assembly, transport and storage processes.

Stochastic input variable are applied on production processes, see Tab. \ref{TableTimes}. An example of a stochastic variable is the installation of a rotor (linking 3 blades with one hub) at sea. For each process a mean duration and a standard deviation of a normal distribution are given instead of a constant value. In this paper the standard deviation has been defined equal to 10\% of the mean time. A sensitivity analysis showed that this parameter has little impact on the results compared to the influence of meteorological data.

Moreover, each process can be affected by production constraints and use different resources (ships, cranes, storage areas). For instance, the installation of the rotor requires a crane and cannot be done before all the tower sections have been assembled. These constraints are different from one process to another depending on the construction sequences and the work breakdown structure of the product.

Transportation processes involve vehicles and routes that were respectively modelled with constant speeds and travelling distance between origins and destinations. Vehicles loading capacities were also included in the model, see Tab. \ref{TableInput}. Additionally, storage processes were modelled defining limited storage capacities.

Furthermore, due to the random nature of the process disturbances, it is not sufficient to rely on the results of a single run. In order to obtain useful results, it is necessary to simulate a scenario several times to identify extreme values and to be able to consider the influence of random variables. So that after running it for a relevant number of times, the user can integrate the output of different results and get the standard deviation diagram, for each Key Performance Indicator (KPI). Similarly, the overall process can be repeated for various starting date within the range of available weather data.

After simulating the construction of the OWF several times, i.e. with various values of the process cycle time distribution and with various weather conditions, one can see where delays occur and what are the possible bottlenecks during the installation process. The planner is interested in the impact of disturbing factors to the entire logistic chain, and thus the total logistic costs. By altering cycle time distributions, adding resources or changing assembly concepts, different installation strategies can be compared.