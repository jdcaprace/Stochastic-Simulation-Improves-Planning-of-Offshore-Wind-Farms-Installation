\section{Justification of the research}
The motivation of the research proposal arose from the discussion on the impacts of climate change and the increasing demands for the use of renewable energies. Wind farms play an important role, as they currently represent the only industrially applicable and economically operable facilities for energy conversion from renewable energy sources.

%The worldwide offshore wind industry is currently facing by new challenges posed by the planned construction of a huge number of new OWFs, see Figure ???.
At the end of 2010, a total of almost 3 GW of offshore wind power capacity had been installed in the European Union. The European Wind Energy Association expects that by 2020 offshore wind power will account for 4 to 4.2\% of Europe's energy demand with an installed capacity of 40 GW, \cite{EWEA2011, Kaldellis2013}.To be economically advantageous, plants have to grow in size ($\geq 5$ MW, i.e. $\sim600-1000$ new wind turbines per year). As a consequence, the entire process of production, mounting and operating offshore wind plants has to be optimized.

Research on OWFs, which until now has been carried out by various institutions, were mainly based on the technical challenges in the design, manufacture, installation and operation of the facilities, \cite{Miller2013, SerranoGonzalez2014, Perveen2014}. Nevertheless, there is only little research that addresses logistical problems in production, installation, maintenance (spare parts logistics) and disassembly (reverse logistics) of offshore facilities, \cite{Scholz2010, Lange2012, COMPIT11, COMPIT12, aitsimulation, thalji2012}.



 %In this way, the end user will have a reliable Key Performance Indicators related to the whole project.

%Actually, no "standard" procedure has been established for transport and installation of offshore wind turbines. The different stakeholders are using different strategies depending of the availability of the offshore vehicles coming from oil and gas sector and civil marine sector. 