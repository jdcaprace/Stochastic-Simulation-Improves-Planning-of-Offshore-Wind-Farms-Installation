\section{Justification of the research}
The motivation of the research proposal arose from the discussion on the impacts of climate change and the increasing demands for the use of renewable energies. Wind farms play an important role, as they currently represent the only industrially applicable and economically operable facilities for energy conversion from renewable energy sources.

%The worldwide offshore wind industry is currently facing by new challenges posed by the planned construction of a huge number of new OWFs, see Figure ???.
At the end of 2010, a total of almost 3 GW of offshore wind power capacity had been installed in the European Union. The European Wind Energy Association expects that by 2020 offshore wind power will account for 4 to 4.2\% of Europe's energy demand with an installed capacity of 40 GW, \cite{EWEA2011, Kaldellis2013}.To be economically advantageous, plants have to grow in size ($\geq 5$ MW, i.e. $\sim600-1000$ new wind turbines per year). As a consequence, the entire process of production, mounting and operating offshore wind plants has to be optimized.

Research on OWFs, which until now has been carried out by various institutions, were mainly based on the technical challenges in the design, manufacture, installation and operation of the facilities, \cite{Miller2013, SerranoGonzalez2014, Perveen2014}. Nevertheless, there is only little research that addresses logistical problems in production, installation, maintenance (spare parts logistics) and disassembly (reverse logistics) of offshore facilities, \cite{Scholz2010, Lange2012, COMPIT11, COMPIT12, aitsimulation, thalji2012}.

There are several publications about installation and maintenance scheduling of OWFs. Typian et al. provided a comparison study of two different mathematical methods for estimating weather downtime and operation times using the Markov Theory and Monte Carlo Simulation, \cite{Tyapin2011}. However this approach is limited to one offshore wind turbine and focuses on operation control. A real scheduling approach was given by \cite{Scholz2010}, who developed a heuristic for the scheduling of offshore installation processes. The current weather situations as well as transport capacity limits of the installation vessel were considered. \cite{ISOPE2012} presented a further going approach for offshore scheduling, which also integrated the inventory control and supply of the installation port.
The problem of offshore maintenance scheduling was treated by \cite{Kovacs2011497}, who developed a MILP, which constituted a module of an integrated framework for condition monitoring, diagnosis and maintenance. The idea of this approach is to find the best time for maintenance operations in relation to performance of the wind turbine and the availability of the service capacities.

The planning and scheduling problem of OWFs considering weather data belongs to the NP-hard problems. It explains why no suitable planning tools for the installation of offshore equipments has been introduced in the literature up to now. Nevertheless, models in literature show some interesting aspects to the installation planning of OWFs using statistical approaches, but do not consider real weather data. The present paper, unlike other research work, presents a DES of the installation of OWFs that is using real weather data. This makes the results of the simulation way more precise and allows to validate other simulations results that are using imprecise seasonal weather forecasts. Since the methodology rely on real weather data, the planning tool may be used during the operational phase of constructing or operating a wind farm. Thus it would be used as a day to day simulator in order to support real-time decisions.

 %In this way, the end user will have a reliable Key Performance Indicators related to the whole project.

%Actually, no "standard" procedure has been established for transport and installation of offshore wind turbines. The different stakeholders are using different strategies depending of the availability of the offshore vehicles coming from oil and gas sector and civil marine sector. 