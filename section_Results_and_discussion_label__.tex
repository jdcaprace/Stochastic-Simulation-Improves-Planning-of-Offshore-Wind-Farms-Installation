\section{Results and discussion}
\label{results}
Outputs are evaluated after each iteration of the DES then averaged when all simulations are completed. Since multiple simulations are run in order to cover all possible situations such as good/bad year for environmental conditions, good/bad performances of the assembly processes, etc., the outputs associated with each scenario is varying. By running a sufficiently large number of simulations, the average results ($\mu$) will converge to a final value and the variability across simulations ($\sigma$) will provide a measure of the uncertainty.

Despite costs might be obtained directly from utilization of the resources, this paper focus the discussions on the \textit{lead time} of the installation of the OWFs, i.e. the time measured in days between the start of pilling activity of the first OWT until the completion of the OWF.

%This is required because making decision only by considering one attribute may lead the OWF integrator to wronf decisions.

The following sections compare and discuss the simulation results of the two different approaches developed.