\textbf{Gap} --
%State the gap - Open Questions. Restrictions and limitations. Nevertheless, key challenges must be addressed in order to find ...., Traditional localization techniques are not well suited for these requirements, ... is cost and energy prohibitigve ..., for many applications it is not sufficiently robust, ... it is limited to ..., One common drawback of the ..., The ... problem has been an open problem of intrest since 1995 ...
In addition, weather plays a critical role in offshore wind energy systems, creating a lot of uncertainties in the logistic system and the foundations design (ground based or floating). Higher wind speeds, waves and the salty air contribute to the harmful environment at sea that significantly reduces the accessibility of an OWFs, \cite{Smit2007}.

There are usually seven components that make up one complete wind turbine: Lower Tower, Upper Tower, Nacelle, Hub, and 3 Blades, see Fig. \ref{fig:WindTurbineComponents}. Whether or not these are pre-assembled, or transported separately (assembly scenario), it has an impact in terms of vessel’s deck space usage, crane lift requirement, and installation capability. The dimensions and weight of the components involved, e.g. blades of more than 120 meters length, make the transport and storage of the components difficult. It will also affect the time necessary to transport and install the turbine components taking into account the suitable weather (time) windows defined by the acceptable wind speed, wave height and sea state. All currently known mounting techniques can only be performed in calm sea. In this context, with a significant wave height of more than 1.5 meter and/or wind speed of 17 m/s at 10 meter of altitude, installation and transport of material at sea will generally be stopped, \cite{aitsimulation}.
