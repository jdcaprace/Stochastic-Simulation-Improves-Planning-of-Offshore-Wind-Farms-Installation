\section{Reviewer comments}
Reviewer #1:
This paper mainly presented a so-called Discrete Event Simulation method to simulate and evaluate the process of the installation of a offshore wind farm with an aim to reduce the cost of offshore wind energy development. However, the paper can be more regarded as an engnieering report than an academic paper. Furthmore, the model built in this paper cannot be extended to be applied in a real project since they only represented a matured engineering process and contained a lot of uncertanties, which cann't be untilized to do the estimation before construction. Therefore, the reviewer will not recommend the paper to be published in RE.


Reviewer #2: This manuscript discusses a simulation framework for the installation phase of an offshore wind farm. After discussing different technological options and installation strategies, and the choices made for the parameters, simulation results for the same project in four different years demonstrate some of the variability due to weather.

The topic is relevant and the paper relatively well written, but it does not seem to have any scientific significance. In particular, the paper does not advance current knowledge. It is nice that such simulations are now possible, but this alone does not constitute reason enough for publishing in the permanent literature. The manuscript would be suitable as a conference paper. I cannot recommend publication in Renewable Energy.

Further major issues that you might want to consider:

1. References / Citations are incomplete. Work by Dinwoodie et al., Hofmann or Scheu et al. on modeling and stochastic simulation of wind park operations & maintenance should be considered, as this is very similar.

2. The model is not described with enough detail. A technical report or another publication should be cited that fully describes the model.

3. L70: "The planning and scheduling problem ... belongs to the NP-hard problems." - Why? Where is this shown? - Needs a reference!

4. L176: Unclear what the time step information (2000-01-01 20:20:20) means - please explain.

5. L177ff: What are the "first" and "second" altitudes mentioned?

6. L204: DES should have 1-2 standard references (textbooks)

7. L234: I understand that with a StdDev of 10 percent the influence of the variability in processes is negligible compared to the role the weather plays, but when I look at the processes in Table 4 I am surprised that the times can be realized so accurately. I would expect that one needs much higher StdDev to model most of these processes with realistic statistics. Please comment.

8. L246: What are the key performance indicators for an offshore wind park that you can calculate with your model?

9. L254ff: Is it possible to run the simulation with more than one vessel?

10. L266: "total duration of the simulation runs ... is about 90 minutes" - you mean wall time?

11. Table 4: What statistical distributions are used here (Gaussians?)

12. l294: "which proves that the weather data has been implemented in a correct fashion." - I do not see a proof here, although it certainly "suggests" this.


In addition, a few minor issues as well:

- L39: "HAS been developed"
- Table 1: "Civil work" - What is this?
- L67: Abbreviation "MILP" should be explained before use
- L80: "Firstly" does not exist
- L154: "In this paper THE installation strategy with the pre-assembly at harbour HAS been considered."
- L175: "These measurements are related to the following parameters:" - unclear, please rephrase
- L189: "has been defined for each resource based on"
- L191: "DOES NOT exceed"
- L197: "the probability OF not having"
- L222: "a DES HAS been developed"
- L223f: "A validation case HAS been developed in partnership with an industrial PARTNER that provided..."
- L230: "Stochastic input variables"
- L247: "for various starting DATES"
- Table 4: "HAS been applied"
- L308: "that can improve planning"
- L318: "This is the STARTING point..."
