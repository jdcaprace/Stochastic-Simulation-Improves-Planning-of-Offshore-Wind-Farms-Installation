
%State the importance of your study
\textbf{Importance} --
In consequence, the development of sustainable installation procedures taking into account long distances and special challenges of maritime transport is required. The simulation of different installation strategies can support the planning process and reduce the risks of assembling OWFs. The model should be able to take into account the effects of weather on the installation cycle, to assess the likelihood of delays of a certain process and to propose alternatives to minimize the effects of these delays. By evaluating different sets of weather data, the over-all installation process can be optimized with respect to shortest installation times and highest robustness of the schedule.

%State the purpose of the paper - Here, we report as new ..., In this paper we describe ..., In this paper we attempt to adress the need to ...
\textbf{Purpose} --
In this paper, a Discrete Event Simulation (DES) of OWFs installation has been developed using weather dependant transport and installation at sea. The originality of the model is to use real offshore weather data in order to quantify accurately the transport and installation risks of a European specific site.Probablistic approach has also been included in the analysis.


ADD reference from Stracklyde

The present paper, unlike other research work, presents a DES of the installation of OWFs that is using real weather data of a specific site. This makes the results of the simulation way more precise and allows to validate other simulations results that are using imprecise seasonal weather forecasts. Since the methodology rely on real weather data, the planning tool may be used during the operational phase of constructing or operating a wind farm. Thus it would be used as a day to day simulator in order to support real-time decisions.


The installation of offshore facilities is very costly and any interruption along its supply chain could cause a big impact on the overall operation. Hence having a well- organized transporting and installation system is very crucial for the offshore industry that can be supported very well with the development of tools taking into account all the variables including distance between ports and installation sites, wind speed, wave height and sea state parameters.

The simulation tools will integrate all the parameters (actors) pertaining in the offshore wind energy development system (contractor, ports, electric companies, various manufacturers, shipping co., etc.), which will be useful to perform a robust multi-criteria simulation to derive comprehensive deployment and installation strategies.

However, no tools are available that enable developers to perform a robust multi-criteria simulation to derive comprehensive deployment and installation strategies. To avoid supply chain bottlenecks and to provide an effective decision support tool, an integrated and comprehensive simulation platform tool of the fixed and floating offshore wind turbine installation (as well as other types of installation), covering the whole lifecycle of an offshore wind park installation that takes into account real weather data, is required. 