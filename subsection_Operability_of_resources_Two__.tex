\subsection{Operability of resources}
Two approaches, i.e. one deterministic and one probabilitic, have been considered in order to analyze the operability of the transport resources. They are differing by the way to use the metocean time series in the DES. However, both approaches are using the \textit{maximum operational wind speed limit} and the \textit{maximum operational wave height limit} presented in Tab. \ref{tab:ressources} as well as the \textit{weather window} define in Tab. \ref{tab:times}.

The deterministic approach has the advantage to perform assessments on historical environmental data which allow the calibration of the simulation model based on past projects but does not allow the scheduling of new projects. Contrarely, the probabilistic approach is designed to be used for planning of new projects as it is using monthly operability probabilities. %These probabilities to work have been computed for each activity considering the workability and weather window restrictions.

\textbf{Deterministic approach} -- In the first approach, the operability of the transport resources is assessed in a deterministic way. The historical environmental data available are consulted during the DES to assess if an activity can be started based on both weather limits and weather window. For instance, the vessels are not allowed to start an operation if either the current wind speed or wave height is higher than the \textit{maximum operational wind speed limit} or the \textit{maximum operational wave height limit} defined in Tab. \ref{tab:ressources}. In addition, the minimum weather window should be longer than the timeframe defined by the respective duration of each activity (Tab. \ref{tab:times}). In this case, the minimum weather window is defined by the consecutive time steps in which wave height and wind speed values are lower than the limiting wave height and wind speed for a specific operation. When the weather is not favourable, the vessel stays at the site until next favourable conditions are met. When the weather window is sufficient enough, the vessel start its next activity. 

\textbf{Probabilistic approach} -- The operability of the transport resources are assessed probabilistically in the second approach. Here, the metocean time series are used to compute an operability probability for each transport resource and assembly activity on a monthly base. In this case, the metocean time series will be parsed by month while a binning routine will segregate the time series with a bin size corresponding to the weather window. Each bin of the month is considered as potential candidates to start operating and thus wind speed limit, wave height limit and weather window are checked similarly to the first approach. In this case, weather window is varied systematically from $1$ to $24\, h$ by step of one hour, wind speed from $5$ to $20\, m/s$ by step of one meter per second and wave height from $0.5$ to $3.5\, m$ by step of $0.25$ meter. As illustration, Tab. \ref{tab:proba} gives an example of the computed probability for September 2000. Considering the example of the upper tower and nacelle installation of an OWT by a jack-up vessel, see Tab. \ref{tab:times} and \ref{tab:ressources}, it can be red as follow. There is a probability of $64.89\, \%$ to encounter environmental conditions corresponding to a minimum weather window of $6\, h$ with a wind speed inferior or equal to $17\, m/s$ and a wave height inferior or equal to $1.25\, m$ during the month of September 2000. Finally, average of the probability for a specific month, e.g. September, is computed by averaging the results for each month along the available years inside the time series.

%evaluate what is the percentage of the days that is favourable to start the operation of the vessel.  ???period???. Then, the probability of operability of the vessel is obtained by counting how many days of the month are favourable to start a specific activity.  ???table???

Later in the DES, triggers are set for each assembly and transport activity using Bernoulli probability mass function given by equation \ref{eqn:bernoudist}. A Bernoulli distribution is the probability distribution of a random variable which takes the value $1$ with success probability of $p$ and the value $0$ with failure probability of $q=1-p$ over possible outcomes $k$. It is also a special case of the binomial distribution where $n=1$.

\begin{equation}
\label{eqn:bernoudist}
f\left(k \vert p \right) = \begin{cases} p & \text{for $k=1$} \\ q=(1-p) & \text{for $k=0$} \end{cases}
\end{equation}

However, the previous explained method can be only applied to independent assembly or transport activities. When there are activities to be carried out one after other using the same resource and without interruption, it requires the implementation of conditional probability. Here, it is the case of the \textit{pile transfer} operation that is conditioned to the end of the \textit{pile transport} operation and the \textit{jacket installation} that is conditioned to the end of the \textit{jacket transport} activity.

In probability theory, \cite{Thalemann2012}, a conditional probability measures the probability of an event given that another event has occurred. Following Kolmogorov definition, given two events $A$ and $B$, the conditional probability of the event $B$ considering that previous event A has occurred is defined as the quotient of the probability of the joint of events $A \cap B$, and the probability of $A$, see equation \ref{eqn:kolmogorov}. Therefore, the probability to have two events $A$ and $B$ occurring successively can be assessed by equation \ref{eqn:finetti}.

It can be exemplified with the case of the \textit{pile transfer} operation that is conditioned to the end of the \textit{pile transport} operation. In DES, first the probability of the \textit{pile transport} operation is assessed then multiplied by the probability of the \textit{pile transfer} operation considering that the \textit{pile transport} operation has occurred.
%For example, considering the pre-pilling phase, the probability of the transport ship to leave the port is first assessed (first activity), then, conditional probability to perform the transfer of piles onto the installation vessel is performed (second activity).

\begin{equation}
\label{eqn:kolmogorov}
P \left(B \vert A \right) = \frac{P \left(A \cap B \right)}{P \left( A \right)}
\end{equation}

\begin{equation}
\label{eqn:finetti}
P \left(A \cap B \right) = P \left( A \right)\, P \left(B \vert A \right)
\end{equation}

%The tree diagram shown in Fig. \ref{fig:condprob} may represent a series of independent events or conditional probabilities. Each node on the diagram represents an event and is associated with the probability of that event. The root node represents the certain event and therefore has probability 1. Each set of sibling nodes represents an exclusive and exhaustive partition of the parent event. The probability associated with a node is the chance of that event occurring after the parent event occurs. The probability that the series of events leading to a particular node will occur is equal to the product of that node and its parents' probabilities.
