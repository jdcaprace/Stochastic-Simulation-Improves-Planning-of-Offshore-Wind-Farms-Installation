Each process can be affected by constraints and can use different resources (ships, cranes, storage areas). For instance, the installation of the rotor requires a crane and cannot be done before all the tower sections have been assembled. These constraints are different from one process to another depending on the construction sequences and the work breakdown structure of the product.

Table \ref{tab:times} shows the summary of the mean times of each manufacturing activities that has been implemented in the logistic chain simulation. These activities are stochastic by nature. Hence, Gauss distribution given by equation \ref{eqn:normaldist} have been applied, where $\mu$ is the mean and $\sigma$ the standard deviation. The standard deviation has been defined equal to 10\% of the mean time. A sensitivity analysis, varying $\sigma$ from 10\% to 100\% by step of 10\%, showed that this parameter is secondary compared to the influence of the metocean data. 

\begin{equation}
\label{eqn:normaldist}
f\left( x \right) = \frac{1}{\sigma \sqrt{2 \pi} } e^{- \frac{1}{2} \left( \frac{x - \mu}{\sigma} \right)^{2}}
\end{equation}

Transportation processes involve vessels and vehicles that were respectively modelled with constant speeds and travelling distance between origins and destinations through distance matrix. Tab. \ref{tab:ressources} give a summary of the quantities, capacities, operational speeds and weather limits of the principal transport resources used in the model. Additionally, storage processes were modelled defining limited storage capacities.

Due to the random nature of the process disturbances, it is not sufficient to rely on the results of a single run. In order to obtain useful results, it is necessary to simulate a scenario several times to identify extreme values and to be able to consider the influence of random variables.

After simulating the construction of the OWF several times, i.e. with various values of the process cycle time distribution and with various weather conditions, one can see where delays occur and what are the possible bottlenecks during the installation process. The planner is interested in the impact of disturbing factors to the entire logistic chain, and thus the total logistic lead time. By altering cycle time distributions, adding resources or changing assembly concepts, different installation strategies can be compared.