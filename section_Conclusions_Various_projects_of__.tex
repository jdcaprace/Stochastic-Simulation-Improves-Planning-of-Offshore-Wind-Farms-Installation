\section{Conclusions}
Various projects of OWFs are actually suffering considerable delays and cost overruns due to a bad evaluation of weather risks and restricted time windows for logistic processes at sea, \cite{TCE12, ISOPE2013}. Logistics costs represent definitively a major risk factor for the offshore wind industry. Better efficiency will be attained improving the predictability and transparency of the logistical processes, both on land and at sea.

The contribution presented in this paper relates the development of a simulation of OWFs installation using both the historical weather time series recorded & probablistic approach that can improve the planning and control of the logistics processes in the wind energy industry.The project lead time results obtained form two approaches have been compared and a very good correlation has been found between them. Results in this study also  indicated that existing tools can be supported by simulation to assess possible disturbances and project risks. With the aid of simulation, different logistics strategies can be tested and compared before starting to implement a new concept in reality so that risks and total costs can be estimated beforehand

The investigated scenario for real weather data of a specific OWF shows that, especially in the final part of the process, which is the transport of the main components to the wind farm and the installation at sea, disturbances due to weather restrictions can lead to an explosion of delays. 

Since the methodology relies on real weather data, the planning tool may be used during the operational phase of constructing or operating a wind farm. Thus it would be used as a day to day simulator in order to support real-time decisions.

This is the staring point of further research. In a following study, statistical weather data will be used in order to be able to make better forecasts of the schedule of OWFs installation. Both methods, relying on statistical and real weather data will be compared and combined in order to reduce as much as possible the effect of disturbances due to weather restrictions.