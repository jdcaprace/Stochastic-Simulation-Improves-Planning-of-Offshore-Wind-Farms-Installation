\begin{abstract}
%Contextualization
\textbf{Contextualization --}
The wind industry is facing new challenges due to planned construction of thousands of offshore wind turbines all around the world.

%Gap - Altough ..., However ...
\textbf{Gap --}
However, with increasing distances from the shore, higher water depth and rising sizes of the plants, the industry has to face the challenge to develop sustainable installation procedures in order to implement projects in the planned cost and time. Important limiting factors for offshore wind farms installation are the weather conditions and installation strategies. Hence, optimal strategies need to take weather conditions into account. 

%Purpose - This paper propose ...
\textbf{Purpose --}
This paper propose an analysis on how weather condition and installation strategies can affect the transport and installation of offshore wind farms.

%Methodology - .... results in ....
\textbf{Methodology --}
A stochastic simulation model has been developed where weather restrictions, distance matrix, vessel characteristics and assembly scenarios are taken into account. The model maps out the logistics chain in the offshore wind industry. A cross validation between an historical weather data model and a probabilistic approaches have been performed.

%Results - We suggest that ..., The results point to the development of ..., These findings provide ...
\textbf{Results --}
The results pointed out a good agreement between the two considered models while a huge risk on the installation time and costs due to the stochastic nature of the weather is highlighted.

%Conclusions - We suggest that the
\textbf{Conclusions --}
We suggest that simulations may improve and reduce the risks on the planning activities of offshore wind farms.

%\keywords{Discrete Event Simulation \and Logistic \and Offshore \and Wind Farm \and Risk \and Installation Strategies \and Probablity}
\end{abstract}
  