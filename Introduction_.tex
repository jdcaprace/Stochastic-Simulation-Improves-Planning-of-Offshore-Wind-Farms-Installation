\begin{abstract}
%Contextualization
\textbf{Contextualization --}
The wind industry is facing new challenges due to planned construction of thousands of offshore wind turbines all around the world.
%Gap - Altough ..., However ...
\textbf{Gap --}
However, with increasing distances from the shore, higher water depth and rising sizes of the plants, the industry has to face the challenge to develop sustainable installation procedures in order to implement projects in the planned cost and time. Important limiting factors for offshore wind farms installation are the weather conditions and installation strategies. Hence, optimal strategies need to take weather conditions into account. 
%Purpose - This paper propose ...
\textbf{Purpose --}
\textit{The purpose of this paper is to carry out an analysis on the effect of weather condition and installation strategies for the transport and installation of wind turbines at sea using a Discrete Event Simulation (DES). }


%Methodology - .... results in ....
\textbf{Methodology --}
\textit{
A stochastic simulation model has been developed where weather restrictions, distance matrix, vessel characteristics and assembly scenarios are taken into account. The model maps out the logistics chain in the offshore wind industry.Both real historical weather data (from a specific offshore site) and probabilistic approaches have been implemented in the analysis.}


%Results - We suggest that ..., The results point to the development of ..., These findings provide ...
\textbf{Results --}
\textit{The results presented in this paper indicated that there is huge risk on the installation time and costs due to the stochastic nature of the weather.  }


%Conclusions - We suggest that the
\textbf{Conclusions --}
\textit{We suggest that simulations may improve and reduce the risks on the planning activities of offshore wind farms.}

\keywords{Discrete Event Simulation \and Logistic \and Offshore \and Wind Farm \and Risk \and Installation Strategies \and Probablity}
% \PACS{PACS code1 \and PACS code2 \and more}
% \subclass{MSC code1 \and MSC code2 \and more}
\end{abstract}
  