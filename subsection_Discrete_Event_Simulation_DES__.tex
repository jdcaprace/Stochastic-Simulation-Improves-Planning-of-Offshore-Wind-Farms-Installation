\subsection{Discrete-Event Simulation -- DES}
Due to the random nature of the process disturbances, it is not sufficient to rely on the results of a single run. In order to obtain useful results, it is necessary to simulate a scenario several times to identify extreme values and to be able to consider the influence of random variables.

After simulating the construction of the OWF several times, i.e. with various values of the process cycle time distribution and with various weather conditions, one can see where delays occur and what are the possible bottlenecks during the installation process. The planner is interested in the impact of disturbing factors to the entire logistic chain, and thus the total logistic lead time. By altering cycle time distributions, adding resources or changing assembly concepts, different installation strategies can be compared.