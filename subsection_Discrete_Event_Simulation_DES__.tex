\subsection{Discrete Event Simulation (DES)}


TODO -- REWORK

%Literature review shows that there have been some significant tangible outcomes in healthcare organizations that adopted Lean principles such as increased patient throughput
%However, many papers identified barriers when implementing Lean management in healthcare organizations such as lack of ownership of proposed processes, skepticism and resistance to change
%In the last few years, discrete-event simulation has been considered as an interesting tool to help improving healthcare services
%Discrete-event simulation has been applied to solve a wide variety of healthcare problems such as patient appointment systems
%Too often, discrete event simulation models have been developed and used by experts to find solutions without involving stakeholders in the development process. Recently, more work has been done on facilitated modeling to involve stakeholders in the development of discrete event simulation models.

This principle is very convenient to map out the logistics chain of the transport and installations of the offshore wind farms since any movements in between have little interest for the simulation itself. One of the major advantages of the DES is that it is possible to integrate the operating rules of each workshop or transportation activity and simulate the complex integration between the different actors (human and material resources, transportation, machinery and tools, etc.).  
The DES has been used in order to map out the logistics chain starting from the producers’ site to the offshore site including loading, unloading, transport, and transfer and installing activities.
