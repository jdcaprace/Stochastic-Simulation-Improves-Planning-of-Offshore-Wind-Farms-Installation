\subsection{Discrete-Event Simulation (DES)}

Outputs are evaluated after each iteration of the DES then averaged when all simulations are completed. Since multiple simulations are run in order to cover all possible situations such as good/bad year for environmental conditions, good/bad performances of the assembly processes, etc., the outputs associated with each scenario is varying. By running a sufficiently large number of simulations, the average results ($\mu$) will converge to a final value and the variability across simulations ($\sigma$) will provide a measure of the uncertainty.
%Due to the random nature of the process disturbances, it is not sufficient to rely on the results of a single run. In order to obtain useful results, it is necessary to simulate a scenario several times to identify extreme values and to be able to measure the influence of uncertainties.

Despite costs might be obtained directly from utilization of the resources, this paper focus the discussions on the \textit{lead time} of the installation of the OWFs, i.e. the time measured in days between the start of pilling activity of the first OWT until the completion of the OWF. By altering cycle time distributions, adding resources or changing assembly concepts, different installation strategies and assembly processes can be compared. This is required because making decision only by considering one attribute may lead the OWF integrator to wrong decisions.

%After simulating the construction of the OWF several times, i.e. with various values of the process cycle time distribution and with various weather conditions, one can see where delays occur and what are the possible bottlenecks during the installation process. The planner is interested in the impact of disturbing factors to the entire logistic chain, and thus the total logistic lead time. 
