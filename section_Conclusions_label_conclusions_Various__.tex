\section{Conclusions}
\label{conclusions}
Various projects of OWFs are actually suffering considerable delays and cost overruns due to a bad evaluation of weather risks and restricted time windows for logistic processes at sea, \cite{TCE12, ISOPE2013}. Logistics costs represent definitively a major risk factor for the offshore wind industry. Better efficiency will be attained improving the predictability and transparency of the logistical processes, both on land and at sea.

The contribution presented in this paper relates the development of a simulation of OWFs installation using probablistic approach and metocean data to improve the planning and control of the logistics processes in the wind energy industry. The project lead time results obtained form two different approaches have been compared and a very good agreement has been found between them. With the aid of simulation, different logistics strategies can be tested and compared before starting to implement a new concept in reality so that risks can be reduced.

Since the methodology relies both metocean time series and probablistic approach, the planning tool may be used during the operational phase of constructing or operating a wind farm. Thus it could be used as a day to day simulator in order to support real-time decisions.
