\subsection{Discrete Event Simulation}
TODO -- REWORK

In the field of simulation, a discrete-event simulation (DES), models the operation of a system as a discrete sequence of events in time. Each event occurs at a particular instant in time and marks a change of state in the system. Between consecutive events, no change in the system is assumed to occur; thus the simulation can directly jump in time from one event to the next. Since discrete-event simulations do not have to simulate every time slice, they can typically run much faster than the corresponding continuous simulation, \cite{Myron1987}, \cite {William1988}.
This principle is very convenient to map out the logistics chain of the transport and installations of the offshore wind farms since any movements in between have little interest for the simulation itself. One of the major advantages of the DES is that it is possible to integrate the operating rules of each workshop or transportation activity and simulate the complex integration between the different actors (human and material resources, transportation, machinery and tools, etc.).  
The discrete event simulation has been used in order to map out the logistics chain starting from the producers’ site to the offshore site including loading, unloading, transport, and transfer and installing activities.
