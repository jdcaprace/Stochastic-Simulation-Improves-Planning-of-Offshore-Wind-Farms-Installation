\section{Challenges of offshore wind turbines installation}
The challenges of installing offshore wind turbines is due to several factors. Firstly, wind turbine are made up of various large and heavy components that are produced in different sites. Secondly, there are various possible installation scenarios depending on infrastructure and transport vessels availabilities. Thirdly, the installation of the offshore wind turbines is strongly influenced by weather conditions.  
Finally, the material process flow can be affected by various disturbances.

\subsection{Wind turbine components}
An offshore wind turbine usually consist of, see Figure \ref{fig:WindTurbineComponents}:

\begin{itemize}
\item A foundation -- that can be of several types, see Figure \ref{fig:Foundations}. Types of foundation will determine the type of equipment required for installation. For instance, a mono pile foundation will require heavy hydraulic hammer works to ram steel pipes with diameters of 4 meters up to 20 meters into the seabed.

\begin{enumerate}
\item[a] -- Mono pile foundations: they can be either concrete or pre-stressed, used for low or mid-level water depths, and having as advantage low levels of noise emission in operation, low maintenance, material availability with large-scale production.
\item[b] -- Tripod foundations: designs tend to rely on technology used by the oil and gas industry. The piles on each end are typically driven into the seabed, used for deeper depths and have not been used on many projects until now.
\item[c] -- Jacket foundations: can be made of a steel framework with pile foundation, used mainly for great level water depths and have the advantages of light weight and high rigidity.
\item[d] -- Gravity foundations: they can be made of restrained steel pile, used mainly for lower level water depths and having the advantage of being a simple and cost-efficient construction for small depths. 
\end{enumerate}

\item Piles -- The piles are used to fix the foundation to the seabed. During this process, a template is used when hammering or vibrating piles into the seabed. Afterwards, the jacket could be lowered to the bottom of the sea, where the spikes fixed at the end of the legs of the jacket fit into the piles.
\item Tower segments -- have the structural role of carrying the top loads to the foundation. They are made from steel sheet rings and stiffeners (longitudinal or circular, used for rigidity purposes) protected against the strong corrosion due to sea water.
\item One nacelle (machine house) -- results from a combination between a steel lattice structure and fiber glass housing. The hydraulic, electrical and electro-mechanic internal components of the nacelle (gearbox, transformers, cooling systems, etc.) are integrated progressively during the construction of the nacelle. It is important to consider that the nacelle is very heavy (125 tons for a 4MW turbine). Together with the rotor, the weight of the nacelle represents a big problem in terms of lifting capacities.
\item One rotor hub -- it corresponds to the mechanical part that joins the 3 blades together with the nacelle.
\item Three blades -- offshore wind turbines blades are made of composite material. At presents wind blades are mainly made of reinforced fiber glass. For very large blades, carbon fibers have been introduced by many manufacturers in order to reduce the structure weight. The size of the blades is increasing proportionally with the increase in power of the wind turbine. For a 3MW wind turbine, the rotor has a weight of around 100 tons and 100 meters diameter.
\end{itemize}

%???? Cable laying ??????

The size of the newest wind turbines is growing quickly in comparisons with previous generations. The latest generations of wind turbines that will soon be installed at sea have a power of 6 MW, which largely outperform the currently installed offshore wind turbines, whose power does not exceed 5 MW. The current trend is to develop high power machines to increase the energy produced by the OWFs, while reducing the number of machines. The size of the turbines significantly increases the difficulty of construction and the risks during assembly at sea. Moreover, as offshore machines increase in size and weight, more manufacturers will be relocated directly to or in the proximity of the port facilities to ease transportation of machines and delivery of components.

Today, the supply chain for offshore wind turbines relies on processes that already exist in onshore wind industry and offshore structures for the gas and oil sector. In the future, the experience gained from OWFs and the need to produce in series for large scale projects should lead to standardization and important optimization in the supply chain processes.

%All these components are produced in different locations and can be assembled either onshore or offshore.

In this paper, one jacket foundation, four piles, 2 tower segments, one nacelle, one rotor hub and three rotor blades per wind turbine have been considered in the simulation.

\subsection{Stochastic weather conditions}
Another aspect concerns the effects of weather uncertainties that have a major importance upon the process of installation of offshore wind turbines, \cite{COMPIT11}. The jacking process of an offshore vessel can only be executed up to a certain wave height and wind speed. Moreover, it is important to note that certain weather parameters can have an impact on certain process parameters. The speed of a ship depends on the wave height or good weather conditions will increase the amount of components that can be loaded on board the vessels. At the moment, the offshore companies plan to install the foundations during the period between autumn and spring, and to assemble the nacelles and rotors during the summer. Because of the seasonal weather changes, it is hard to predict the efficiency and associated risks of the installation strategies.

Weather predictions and numerical weather forecasts can be calculated with different models. However, the reliable weather predictions are mostly provided for a period of approximately a maximum of 14 days, \cite{hinnenthal2007}. This is obviously not appropriated for a long-term scheduling.

Beyond that, seasonal weather forecasts exist. They are based on historical data and give ideas such as average wind speeds, temperatures and rain falls in these periods. Based on these long-term forecasts, it can be predicted that in winter the probability of days with good weather conditions is significantly less than in summer. Nevertheless, experiences of OWFs installation show that the reliability of this kind of long-term forecast is not enough for planning. Therefore, accurate forecast (short term) should be used in order to get enough accuracy.

In order to overcome these issues and to consider that weather conditions can vary significantly from one offshore site to another, the model presented in this paper is using real offshore weather data measured every 10 minutes for the last decade.  
%Aggregation of the data will be possible per hour and per day.
These measurements are related to the following parameters:
\begin{itemize}
\item Time step [2000-01-01 20:20:20];
\item Average wind speed at the first altitude in m/s;
\item Average wind speed at the second altitude in m/s;
\item Average wave height in meters.
\end{itemize}

Weather restrictions can be defined for resources, processes and components in the model. For example, seagoing specialized vessels like jack-up barges or other installation units are characterized by weather constraints, see Table \ref{TableWorkability}. Depending on wind force or wave height, or both, the vessel might not be allowed to leave the port and start its journey to the OWF although the loading process is completed. Having arrived at its destination at sea, processes like the installation of each component of a wind turbine can be delayed too.

Additionally to these criteria, a "time window" concept was defined for the utilization of the resources. It is defined by the amount of time in which the weather parameters will allow the process to occur under safe weather conditions. This time window was implemented for each transport resource shown in Table \ref{TableWorkability}.

A "workability" criterion has been defined for each resources based on measured weather data and criteria defined in Table \ref{TableWorkability}. This parameter verifies whether or not an activity can be started at a certain date. If the weather parameter (wind speed or wave height) doesn't exceed the working limits during the defined time window, the activity can be started. Otherwise, the activity will be delayed until favourable weather conditions are appearing.

\begin{table}[!hbp]
\caption{Criteria giving the weather conditions over which it is no longer possible to use certain resources}
\begin{center}
\include{TableWorkability}
\end{center}
\label{TableWorkability}
\end{table}

\subsection{Disturbance of the material process flow}
Other disturbances than the weather conditions have been identified in the material flow process. These disturbances will be a source of additional delays in the supply chain. Amongst these disturbances we find the probability of resource failure, the probability to have missing materials and components, the probability not having enough workforce, the probability to reach the maximum level of storage surface in the port, etc. In the simulation, these probabilities are expressed as stochastic distribution in order to take into account the incertitude of the input parameters. In that way, the risk on the variance of the schedule can be evaluated.
