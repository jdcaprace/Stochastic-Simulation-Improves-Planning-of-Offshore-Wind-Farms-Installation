\section{Results and discussion}
This section is dedicated to analyze the simulation results obtained after running the developed simulation model with two different input data model. Both weather historical time series and probabilistic approach are considered. Then results are compared and discussed.

The time series of wind speed and wave height has been used in the simulation model to quantify the total completion time of an OWF including 60 OWT. Wind speed was measured every 10 minutes at two different altitudes ( ??? and ???) and wave height every 30 minutes. Data have been recorded from 1995 and 2008. In order to have coherence between the time intervals, wave height measurements has been (??? linearly) interpolated every 10 minutes.

A new simulation run is created for each month from 1995 until to 2006. The starting date of the project has been assumed to be the first day of each month. Thus, it represents 12 simulations per month what is mean 144 simulations in total. Therefore, the average lead time ($\mu$) and standard deviation ($\sigma$) is deduced from the 12 points available for each month.