%Show the state of the art - Recently they have been significant advances in ..., Previous research in this area has focused on ...
\textbf{State of the art} --
Research on OWFs, which until now has been carried out by various institutions, were mainly based on the technical challenges in the design and manufacture of the facilities, \cite{Miller2013}, \cite{SerranoGonzalez2014}, \cite{Perveen2014}. Nevertheless, there is only little research that addresses logistical problems in production, installation, operation, maintenance (spare parts logistics) and disassembly (reverse logistics) of offshore facilities, \cite{Scholz2010}, \cite{Lange2012}, \cite{COMPIT11}, \cite{COMPIT12}, \cite{aitsimulation}, \cite{thalji2012}.

Current practices to establish the long term planning to build an OWF is based on statistical data. However, statistical data is insufficient for the long term planning process since it does not account for the weather factors which have major effect on all offshore operations. According to today’s knowledge, there is no specific tool available to perform simulation to explicitly consider the scattering (randomness) of many parameters (e.g. meteocean parameters, availability ratio of mono hull vessels and jack-ups) and to simulate various scenarios regarding the deployment strategy of offshore wind systems. This causes a large uncertainty on the timeframe of the installation process so the date of delivery cannot be predicted accurately. Therefore, the risk can only be controlled by taking some safety margin which results in higher deployment and installation costs and/or a longer duration, which are not compatible with the development of large offshore wind parks.

The planning and scheduling problem of OWFs considering weather data belongs to the NP-hard problems (Non-deterministic Polynomial-time hard), \cite{leeuwen1990}. It explains why no suitable planning tools for the installation of OWFs including metocean data has been introduced in the literature up to now.

Simulation is a modelling tool widely used in Operational Research (OR), \cite{pidd2005computer}. One of the most common approach is Discrete-Event Simulation (DES). It started and evolved with the advent of computers, \cite{Myron1987}, \cite {William1988}, \cite{robinson2005}. DES represents individual entities that move through a series of queues and activities at discrete points in time. Each event occurs at a particular instant in time and marks a change of state in the system. Between consecutive events, no change in the system is assumed to occur; thus the simulation can directly jump in time from one event to the next. Since discrete-event simulations do not have to simulate every time slice, they can typically run much faster than the corresponding continuous simulation.

Models are generally stochastic in nature. The issues on how to define probability distributions for activity duration trough sample data in manufacturing field have been widely studied \cite{fente2000defining, Maio2000}. The most difficult aspect of duration assessment is gathering data of sufficient quality, quantity and variety. Moreover, impact of activities-duration uncertainties characterizing the real-world due to factor such as weather condition are poorly studied.

DES has been traditionally used in the manufacturing sector \cite{KhedriLiraviasl20151490}, \cite{Yeong2014}, while recently it has been increasingly used in the service sector including applications in airports, call centers, fast food restaurants, banks, health care \cite{Chemweno201445} and business processes \cite{Khodyrev2014322}.

Literature review shows that there have been some significant tangible outcomes in simulation installation of OWFs that adopted probabilistic approaches. \cite{Tyapin2011} provided a comparison study of two different mathematical methods for estimating weather downtime and operation times using the Markov Theory and Monte Carlo Simulation. However this approach is limited to one OWT and focuses on operation control. A real scheduling approach was given by \cite{Scholz2010}, who developed a heuristic for the scheduling of offshore installation processes. The current weather situations as well as transport capacity limits of the installation vessel were considered. \cite{ISOPE2012} presented a further going approach for offshore scheduling, which also integrated the inventory control and supply of the installation port.

Recently they have been significant advances in Operation \& Maintenance (O\&M) simulations, while in contrast, few publications address the logistical problem of manufacturing and installation of OWFs. The problem of offshore maintenance scheduling was first treated by \cite{Kovacs2011497}, who developed a Mixed-Linear Integer Programming (MILP), which constituted a module of an integrated framework for condition monitoring, diagnosis and maintenance. The idea of this approach is to find the best time for maintenance operations in relation to performance of the wind turbine and the availability of the service capacities. Later, \cite{scheu2012} and \cite{Hagen2013} developed a multivariate stochastic weather models in order to generate sea state time series based on observed time series or historical data and validated a simulation of O\&M applied to OWFs.

\cite{dinwoodie2013} developed an econometric O\&M model to determine where different operational choices represent the cost optimal solution. The sensitivity of operational strategies to OWFs size, failure rate of major components and weather conditions have been examined. A multivariate Auto-Regressive climate model were used. This methodology maintains seasonality and correlation between wind and wave time series. However, the model fail to capture outliers and data behavior over 17 $m/s$. \cite{Hofmann2014} pointed out that simulation helps to quantify the cost of O\&M and also indicated that larger wind turbines can lead to lower O\&M costs. It was also concluded that higher failure rates and maintenance durations will quite fast counterbalance the benefits of larger wind turbines. Already a simultaneous increase of failure rates and maintenance durations by 25 \% will lead to higher O\&M cost for a wind farm with 10 MW wind turbines compared to a 5 MW turbine OWF. 
