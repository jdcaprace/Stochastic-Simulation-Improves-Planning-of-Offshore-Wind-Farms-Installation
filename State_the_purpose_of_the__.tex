%State the purpose of the paper - Here, we report as new ..., In this paper we describe ..., In this paper we attempt to adress the need to ...
\textbf{Purpose} --
This paper presents the development of a DES of OWFs installation coupled to metocean data. This principle is very convenient to map out the logistics chain of the transport and installations of the offshore wind farms since any movements in between have little interest for the simulation itself. One of the major advantages of the DES is that it is possible to integrate the operating rules of each workshop or transportation activity and simulate the complex integration between the different actors (human and material resources, transportation, machinery and tools, etc.). 

The development presented here take into account all the variables including distance between ports and installation sites as well as sea state parameters. A deterministic and a probabilistic metocean data module has been compared. Both models are based on time series and layout configuration of the Thorntonbank OWF located at 30 km of the Belgian coast (latitude 51° 38′ 39″ North, longitude 2° 55′ 38″ East), in water ranging from 12 to 27 meters deep, see Fig \ref{fig:map}.

The model is able to take into account the effects of weather on the installation cycle, to assess the likelihood of delays of a certain process and to propose alternatives to minimize the effects of these delays. By evaluating different sets of weather data, the overall installation process can be optimized with respect to shortest installation times and highest robustness of the schedule.

%The present paper, unlike other research work, presents a DES of the installation of OWFs that is using real weather data of a specific site. This makes the results of the simulation way more precise and allows to validate other simulations results that are using imprecise seasonal weather forecasts. Since the methodology rely on real weather data, the planning tool may be used during the operational phase of constructing or operating a wind farm. Thus it would be used as a day to day simulator in order to support real-time decisions.

%The DES has been used in order to map out the logistics chain starting from the producers’ site to the offshore site including loading, unloading, transport, transfer and installing activities.
