%State the purpose of the paper - Here, we report as new ..., In this paper we describe ..., In this paper we attempt to adress the need to ...
\textbf{Purpose} --
This paper presents the development of a Discrete Event Simulation (DES) of OWFs installation. An historical weather data model and a probabilistic approaches have been cross validated. Both models are based on time series and layout configuration of the Thornton bank OWF located at 30 km of the Belgian coast (latitude 51° 38′ 39″ Nord, longitude 2° 55′ 38″ Est), in water ranging from 12 to 27 meters deep.

A cross validation between an historical weather data model and a probabilistic approaches have been performed.

that use weather dependant transport and installation at sea. 

The originality of the model is to use real offshore weather data in order to quantify accurately the transport and installation risks of a European specific site.Probablistic approach has also been included in the analysis.


The present paper, unlike other research work, presents a DES of the installation of OWFs that is using real weather data of a specific site. This makes the results of the simulation way more precise and allows to validate other simulations results that are using imprecise seasonal weather forecasts. Since the methodology rely on real weather data, the planning tool may be used during the operational phase of constructing or operating a wind farm. Thus it would be used as a day to day simulator in order to support real-time decisions.

that can be supported very well with the development of tools taking into account all the variables including distance between ports and installation sites, wind speed, wave height and sea state parameters.

The simulation tools will integrate all the parameters (actors) pertaining in the offshore wind energy development system (contractor, ports, electric companies, various manufacturers, shipping co., etc.), which will be useful to perform a robust multi-criteria simulation to derive comprehensive deployment and installation strategies.

The model should be able to take into account the effects of weather on the installation cycle, to assess the likelihood of delays of a certain process and to propose alternatives to minimize the effects of these delays. By evaluating different sets of weather data, the over-all installation process can be optimized with respect to shortest installation times and highest robustness of the schedule.


Electricity production started in early 2009, with a capacity of 30 MW.[1] The capacity is expected to be increased gradually to 300 MW in 2015.