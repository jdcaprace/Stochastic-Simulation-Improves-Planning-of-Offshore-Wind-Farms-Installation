\subsection{Stochastic weather conditions}
Another aspect concerns the effects of weather uncertainties that have a major importance upon the process of installation of OFTs, \cite{COMPIT11}. For instance, the jacking process of an offshore vessel can only be executed up to a certain wave height and wind speed. At the moment, the offshore companies plan to install the foundations between autumn and spring, and to assemble the nacelles and rotors during the summer. Because of the seasonal weather changes, it is hard to predict the efficiency and associated risks of the installation strategies.

Weather predictions and numerical weather forecasts can be calculated with different models. However, the reliable weather predictions are mostly provided for a period of approximately a maximum of 14 days, \cite{hinnenthal2007}. This is obviously not appropriated for a long-term scheduling.

Weather restrictions can be defined for resources and processes in the model. For example, seagoing specialized vessels are characterized by weather constraints, see Table \ref{tab:ressources}. Depending on wind force or wave height, or both, the vessel might not be allowed to leave the port and start its journey to the OWF although the loading process is completed. Having arrived at its destination at sea, processes like the installation of each component of a wind turbine can be delayed too.

Therefore, two restrictions can be defined in order to check if an activity can be fulfilled at sea. First, the workability that represent the upper values of wind speed and wave height above which operations should be interrupted completely. Second, the time-window that represent the time frame for which operation can be pursued without interruption.  If the weather parameter (wind speed or wave height) does not exceed the workability limits during the defined time window, the activity can be started. Otherwise, the activity will be delayed until favorable weather conditions are appearing.

