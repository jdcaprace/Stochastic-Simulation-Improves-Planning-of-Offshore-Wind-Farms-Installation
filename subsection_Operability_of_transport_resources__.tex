\subsection{Operability of transport resources}
Two approaches have been considered in order to analyze the operability of the transport resources. They are differing by the way to use the metocean time series in the DES. However, both approaches are using the \textit{maximum operational wind speed limit} and the \textit{maximum operational wave height limit} presented in Tab. \ref{tab:ressources}.

\textbf{Deterministic approach} -- In the first approach, the operability of the transport resources are assessed in a deterministic way. The historical environmental data available are consulted during the DES to assess if an activity can be started based on both weather limits and weather window. For instance, the vessels are not allowed to start an operation if either the current wind speed or wave height is higher than the \textit{maximum operational wind speed limit} or the \textit{maximum operational wave height limit} defined in Tab. \ref{tab:ressources}. In addition, the minimum weather window should be longer than the timeframe defined in Tab. ???. In this case, the minimum weather window is defined by the consecutive time steps in which wave height and wind speed values are lower than the limiting wave height and wind speed for a specific operation. If the minimum weather window is shorter than the specified limit, the vessel stays at the site until next favourable conditions are met. When the minimum weather window is sufficient enough, the vessel start its next activity. 

\textbf{Probabilistic approach}

The operability of the transport resources are assessed probabilistically in the second approach. Here, the metocean time series are used to compute an operability probability for each transport resource and assembly activity on a monthly base. In this case, the metocean time series will be parsed by month while a routine will evaluate what is the percentage of the days that is favourable to start the operation of the vessel. Each day of the month is considered as potential candidates to start operating and thus wind speed limit, wave height limit and weather window are checked similarly to the first approach. ???period???. Then, the probability of operability of the vessel is obtained by counting how many days of the month are favourable to start a specific activity. Finally, average of the probability for a specific month is computed by averaging the results for each available year inside the time series. ???table???

Later in the DES, triggers are set using Bernoulli probability mass function given by equation \ref{eqn:bernoudist}. A Bernoulli distribution is the probability distribution of a random variable which takes the value $1$ with success probability of $p$ and the value $0$ with failure probability of $q=1-p$.  It is also a special case of the binomial distribution where $n=1$.

\begin{equation}
\label{eqn:bernoudist}
f\left(k \vert p \right) = \begin{cases} p & \text{for} k=1 \\ q=(1-p) & \text{for} k=0 \end{cases}
\end{equation}



%I wanna model Bernoulli machine. The basic idea of Bernoulli machine is that the machine has p probability up and 1-p probability down no matter which state (up or down) the machine was at the previous time point. In other words, at the beginning of each cycle time (the machine can be either up or down), the machine up/down state will be determined by the given p.
%My question is how can I set this up in the Failure tab of the machine? I know Bernoulli machine corresponds to Binomial distribution with n=1, and a given p.



Approach 1 has the advantage to perform assessments on historical data which allow the calibration of the simulation model based on past projects but does not allow the scheduling of new projects. Contrarely, approach 2 is designed to be used for planning of new projects as it is using working probabilities computed on monthly base. These probabilities to work have been computed for each activity considering the workability and weather window restrictions.



If there are two activities to be carried out one after another without interruption, it requires the implementation of conditional probability.  
In probability theory, \cite{Thalemann2012}, a conditional probability measures the probability of an event given that another event has occurred. Following Kolmogorov definition, given two events A and B, the conditional probability of A given B is defined as the quotient of the probability of the joint of events A and B, and the probability of B, see equation \ref{eqn:kolmogorov}.

\begin{equation}
\label{eqn:kolmogorov}
P \left(A \vert B \right) = \frac{P \left(A \cap B \right)}{P \left( B \right)}
\end{equation}

The tree diagram shown in Fig. \ref{fig:condprob} may represent a series of independent events or conditional probabilities. Each node on the diagram represents an event and is associated with the probability of that event. The root node represents the certain event and therefore has probability 1. Each set of sibling nodes represents an exclusive and exhaustive partition of the parent event. The probability associated with a node is the chance of that event occurring after the parent event occurs. The probability that the series of events leading to a particular node will occur is equal to the product of that node and its parents' probabilities.

In this paper, the upper branch of the tree is used in order to compute the workability of two successive activities. For example, considering the pre-pilling phase, the probability of the transport ship to leave the port is first assessed (first activity), then, conditional probability to perform the transfer of piles onto the instllation vessel is performed (second activity).