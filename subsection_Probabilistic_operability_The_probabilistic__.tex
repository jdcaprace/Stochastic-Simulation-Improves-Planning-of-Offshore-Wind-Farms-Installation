\subsection{Probabilistic operability}
The probabilistic approach involves computing of the monthly conditional probability to operate the resources (jack-up, cranes, etc.). As mentioned before, a method is triggered based on weather criteria, weather window and associated probability. It returns a true or false value whether to proceed to the next activity or to wait until the next time frame of good weather. A loop iterates until the result is true which gives a green light to carry out the next activity (sailing, installing, transferring piles, etc.). The time elapsed until the iteration return a true result is considered as a waiting time. At each iteration of the DES, new random numbers are set-up which change the sequences of the binary values and therefore the lead time of the processes.

Fig. \ref{fig:iterations} represent the lead time of the installation of 60 OWTs along 400 iterations. As the result is highly stochastic, testing the convergence of the output is required. Fig. \ref{fig:convergence} presents the convergence of the lead time related to the installation of 60 OWTs. It is observed that the accumulated mean value tend to converge roughly after 250 iterations, i.e. with a variation of less than one day per iteration.

The lead time obtained from the two approaches has been compared in Fig. \ref{fig:comparison}. It is observed that the two approaches presents in general a good agreement. However, a discrepancy of about 25 days is observed during autumn and winter seasons. The lead time is under estimated by the probabilistic approach because this model is not able to reproduce exceptional conditions (extreme).

%????? Show the percentage of each activity account for in a graph ????????????