\subsection{Environmental conditions}
Another aspect concerns the effects of weather uncertainties that have a major importance upon the process of installation of OWTs, \cite{COMPIT11}. For instance, the jacking process of an offshore vessel can only be executed up to a certain wave height and wind speed. At the moment, the offshore companies plan to install the foundations between autumn and spring, and to assemble the nacelles and rotors during the summer. Because of the seasonal weather changes, it is hard to predict the efficiency and associated risks of the installation strategies.

Weather predictions and numerical weather forecasts can be calculated with different models. However, the reliable weather predictions are mostly provided for a period of approximately a maximum of 14 days, \cite{Hinnenthal_2010}. This is obviously not appropriated for a long-term scheduling.

Weather criterion can be defined for resources and processes in the model. For example, seagoing specialized vessels are characterized by weather constraints, see Table \ref{tab:ressources}. Depending on wind force or wave height, or both, the vessel might not be allowed to leave the port and start its journey to the OWF although the loading process is completed. Having arrived at its destination at sea, processes like the installation of each component of a wind turbine can be delayed too.

Therefore, two restrictions can be defined in order to check if an activity can be fulfilled at sea. First, the \textit{weather-limit} that represent the upper values of wind speed and wave height above which operations should be interrupted completely. Second, the \textit{weather-window} that represent the time frame for which operation can be pursued without interruption.  If the weather parameter (wind speed or wave height) does not exceed the weather limits during the defined weather window, the \textit{workability} is set to true, and the activity can be started. Otherwise, the activity will be delayed until favorable weather conditions are appearing.

Metocean time series of wind speed and wave height has been used in the simulation model to quantify the total completion time of an OWF. Wind speed was measured every 10 minutes at 10 meters of altitude in two perpendicular directions and wave height every 30 minutes. Data have been recorded from January 1995 and December 2008.
%In order to have coherence between the time intervals, wave height measurements has been linearly interpolated every 10 minutes.
Fig. \ref{fig:windwave} shows the mean ($\mu$) and standard deviation ($\sigma$) of wind speed and wave height time series.
Since the wind speed observations are measured from a single altitude, the wind profile power law relationship should be used to interpolate these value to the hub level. Equation \ref{eqn:windprofile} is used to estimate the wind speed $u$ at height $z$, when $u_{r}$ is the known wind speed at a reference height $z_{r}$, \cite{Justus_1976, 1978Peterson}. The exponent $\alpha$ is an empirically derived coefficient that varies depending on the stability of the atmosphere, temperature, pressure, humidity, time of the day and terrain roughness. Here this coefficient is taken equal to $0.1$, value recommended over open water surfaces, \cite{WindEnergy2010}.

\begin{equation}
\label{eqn:windprofile}
u = u_{r} \left( \frac{z}{z_r} \right)^{\alpha}
\end{equation}