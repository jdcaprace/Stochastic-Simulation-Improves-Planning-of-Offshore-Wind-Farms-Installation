\subsection{Installation strategies}
The installation scheduling of OWFs is strongly dependent on the current weather conditions at sea. Accordingly, different attempts exist and they offer options for an optimized scheduling, \cite{Thalemann2012}. Three different installation strategies can be distinguished:

\begin{enumerate}
\item Pre-assembly at harbor -- Turbines, substructures and towers are transported by trucks and/or ships to a support harbor close to the wind farm. Preparation and pre-assembly of different components are carried out at this point and afterwards, the structures are transported and installed at the offshore site by the installation vessel, e.g. a jack-up vessel.
\item Assembly Offshore -- In this strategy, feeder vessels supply an offshore jack-up vessel to the installation site with the pre-installed single components. Assembly and installation operations are then performed from this structure. The main advantage of this method is that the installation vessel does not need to be used for transport. However, an extra loading operation has to be implemented in order to load the feeder vessels or barges.
\item Manufacture and pre-assembly at harbor -- In this approach, the pre-assembled turbines are directly shipped from the manufacturers to the site. Components are transported using high speed jack-up vessels.
\end{enumerate}

The choice of one strategy will depend on the economical balance between the number and type of ships to be used, the distance to shore and the risk involved due to different operations. In this paper, installation strategy with the pre-assembly at harbor is considered for the OFT components (blades, hubs, tower sections and nacelles). However, piles and jacket foundations are directly transported from production site to the OWF. 