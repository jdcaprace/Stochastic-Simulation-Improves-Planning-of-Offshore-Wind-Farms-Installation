\subsection{Deterministic operability}
By successively altering the starting day of the simulation amongst the time series available, i.e. from 1995 to 2006, 144 simulations scenarios have been generated. The simulation starting date of the project has been assumed to be the first day of each month, e.g. 'Januray 1th 1995', 'February 1th 1995', etc. In order to cover additional variations due to randomization of the assembly processes 25 iterations have been set in the DES for each scenario with a standard deviation of the assembly processes set to 10\% of the mean defined in Tab. \ref{tab:times}. Therefore, the average lead time ($\mu$) and standard deviation ($\sigma$) can be deduced for each month.

Fig. \ref{fig:comparison} shows the influence of the starting date of the project on the mean lead time ($\mu$) of the installation of 60 OWTs measured in days. Shaded areas show the standard deviation limits ($\sigma$), i.e. $\mu + \sigma$ and $\mu - \sigma$. It is observed that the standard deviation is presenting a high variability along the year with the minimum value during the summer in August (50 days) and the maximum value at beginning of Autumn in October (94 days). That gives an insight how the weather plays a significant role in the offshore wind turbine installation operations.

Fig. \ref{fig:gantt} presents the best and the worst schedule generated by the model respectively for years 2004 and 1999 with a starting date set to April 1th. In this figure white color means idle time, red color means waiting time, black color means working time, light gray color means transport time and blue color means loading or unloading time. It is observed that the lead time of the year 1999 is almost double of the year 2004. This is mainly due caused by the weather window that is too small to start to install the jacket foundations during the winter 1999. Indeed, the large red bar indicate that the vessel waited for favorable environmental conditions during almost 4 months. However, during year 2004, the project has been completed in only one year. In order to mitigate this risk several options are available such as splitting the project in smaller number of OWTs, changing the charter period of the jack-up or studying the option to buy one.

%????? The operability of offshore transport and installation vessels is relatively low therefore the activities cannot be performed due to rough weather by ????\% Prepara a table or better afigure???? The ???? account for ????. It is important to highlight the fact that installation activities of the OWTs account for only ???\% of the total.